\chapter*{Abstract}

This master's thesis deals with the extent to which individual player performance can be predicted with the help of machine learning models. To answer this question, the current state of research in the field of sports analytics is first conducted through a literature review. Based on the resulting findings, an application is developed that automatically obtains, processes, and stores data from various sources. With the help of this data, different machine learning models are compared based on the CRISP-DM cycle. The influence of betting odds on the accuracy of the models is examined separately by dividing the models into a baseline group without betting odds and a treatment group with betting odds. These investigations show that betting odds improve the predictions of the models slightly. In comparing the machine learning models, an optimised random forest model achieved the most accurate forecasts and earned prizes in the experiment conducted on the test data set. The developed application is extended by this model. It is thus able to predict the individual player performance and send a line-up for the upcoming match day to the user based on these predictions.

\clearpage \chapter*{Abstract German}

Die vorliegende Masterarbeit beschäftigt sich mit der Frage, inwieweit, mit Hilfe von Machine Learning Modellen, individuelle Spielerleistungen vorhergesagt werden können. Um diese Frage beantworten zu können, wird zunächst der aktuelle Forschungsstand im Themenbereich des Sports Analytics zusammengefasst. Basierend auf den daraus hervorgehenden Erkenntnissen wird eine Applikation entwickelt, welche automatisiert Daten aus verschiedenen Quellen bezieht, verarbeitet und abspeichert. Mit Hilfe dieser Daten werden anschließend verschiedene Modelle des Machine Learnings, in Anlehnung an den CRISP-DM Kreislauf, miteinander verglichen. Dabei wird der Einfluss von Wettqouten auf die Genauigkeit der Modelle gesondert untersucht, in dem die Modelle in eine Basisgruppe ohne die Wettqouten und in eine manipulierte Gruppe mit Wettqouten eingeteilt werden. Aus diesen Untersuchungen geht hervor, dass Wettqouten die Vorhersagen der Modelle leicht verbessern. Im Vergleich der Machine Learning Modelle erzielte ein optimiertes Random Forest Modell die genausten Vorhersagen und konnte im durchgeführten Experiment auf dem Testdatensatz Gewinne erzielen. Die entwickelte Anwendung wird um dieses Modell erweitert und ist dadurch in der Lage, zunächst die individuellen Spielerleistungen zu prognostizieren und auf Basis dieser Vorhersagen eine Aufstellung für den kommenden Spieltag an den Nutzer zu verschicken.