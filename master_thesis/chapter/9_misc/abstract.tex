\chapter*{Abstract}

This master's thesis deals with the extent to which individual player performance can be predicted with the help of machine learning models. To answer this question, first, the current state of research in the field of sports analytics is outlined via a literature review. Based on the resulting findings, an application is developed that automatically obtains, processes, and stores data from various sources. With the help of this data, different machine learning models are trained and compared based on the CRISP-DM cycle. The influence of betting odds on the accuracy of the models is examined separately by dividing the models into a baseline group without betting odds and a treatment group with betting odds. This investigation shows that betting odds improve the predictions of the models slightly. While comparing the machine learning models, an optimised random forest model achieved the most accurate forecasts and earned prizes in the experiment conducted on the test data set. The developed application is extended by this model. It is thus able to predict the individual player performance and send a line-up for the upcoming match day to the user based on these predictions.

\clearpage \chapter*{Abstract German}

Die vorliegende Masterarbeit beschäftigt sich mit der Frage, inwieweit, mit Hilfe von Machine Learning Modellen, individuelle Spielerleistungen vorhergesagt werden können. Um diese Frage beantworten zu können, wird zunächst der aktuelle Forschungsstand im Themenbereich des Sports Analytics zusammengefasst. Basierend auf den daraus hervorgehenden Erkenntnissen wird eine Applikation entwickelt, welche automatisiert Daten aus verschiedenen Quellen bezieht, verarbeitet und abspeichert. Mit Hilfe dieser Daten werden anschließend verschiedene Modelle des Machine Learnings, in Anlehnung an den CRISP-DM Kreislauf, miteinander verglichen. Dabei wird der Einfluss von Wettqouten auf die Genauigkeit der Modelle gesondert untersucht, in dem die Modelle in eine Basisgruppe ohne die Wettqouten und in eine manipulierte Gruppe mit Wettqouten eingeteilt werden. Aus diesen Untersuchungen geht hervor, dass Wettqouten die Vorhersagen der Modelle leicht verbessern. Im Vergleich der Machine Learning Modelle erzielte ein optimiertes Random Forest Modell die genausten Vorhersagen und konnte im durchgeführten Experiment auf dem Testdatensatz Gewinne erzielen. Die entwickelte Anwendung wird um dieses Modell erweitert und ist dadurch in der Lage, zunächst die individuellen Spielerleistungen zu prognostizieren und auf Basis dieser Vorhersagen eine Aufstellung für den kommenden Spieltag an den Nutzer zu verschicken.

\clearpage \section*{Danksagung}

\indent An dieser Stelle möchte ich mich bei all denjenigen bedanken, die es mir ermöglicht haben, diese Masterarbeit heute einreichen zu können.

Zunächst möchte ich mich bei meinen Betreuern Prof. Dr. Martin Spott und Dr. Sven Willrich bedanken. Dabei möchte ich die Wortwahl Betreuer besonders hervorheben, da Sie nicht nur die Arbeit begutachtet und benotet haben, sondern darüber hinaus mir stets beiseite gestanden und meine Fragen beantwortet haben, um mich schlussendlich an den nötigen Stellen auf den richtigen Weg zu weisen. 

Doch um eine Masterarbeit abgeben zu können, Bedarf es im Vorfeld auch ein fünf-jähriges Studium. Ich möchte mich deshalb bei allen Professoren und Kommilitonen bedanken die mich auf diesem Weg begleitet haben. Dabei möchte ich mich besonders bei fünf Kommilitonen bedanken: Julien, Laurids, Stefan, Timo und Chris, die über diese Zeit weit aus mehr geworden sind: beste Freunde. Ich bin froh euch fünf kennengelernt zu haben, denn mit euch wurde jede noch so zähe Aufgabe schaffbar.
Ein besonderer Dank gilt dabei dir, Laurids, da wir gemeinsam direkt nach dem Bachelor den Master angehängt haben und keine Gruppenarbeit ohne den Anderen absolvierten. Ich werde unsere gemeinsamen All-Nighter kurz vor Abgabefrist vermissen, sowie deinen stets kühlen und analytischen Kopf. 

Des weiteren bedanke ich mich bei meiner Familie, allen voran meiner Mutter, die stets ein offenes Ohr für mich hat, sowie meinem Bruder Danilo, der mich vor sowie während des Studiums beriet. Ohne euch hätte ich nicht mit diesem Studium angefangen.
Abschließend möchte ich mich bei meiner Freundin Friedi und ihrer Familie bedanken. Danke Friedi, dass du immer für mich da bist, mir gerade in schwierigen Zeiten den nötigen Rückhalt gibst und mich motivierst, niemals aufzugeben.

\vspace{15pt}

{\parindent0pt

Ich möchte diese Arbeit meinem Vater widmen.

\begin{flushright}
Vielen Dank Euch Allen!

Euer Jakob

Leipzig, 08.11.2021
\end{flushright}
}