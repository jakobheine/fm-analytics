\chapter{Implementation}

As already mentioned in chapter 2, the task Fantasy League managers face is to set up an optimal line-up. To find out this optimal set-up, indirect anticipation of the future is always useful. As can be seen from the literature review in \autoref{chap:literature_review}, many attempts have already been made to make this anticipation no longer manual and based on random factors. Quite the contrary, through the combination of big data and sport \parencite[][cf.]{rein_big_2016}, it is now possible to use the data collected in the past to make automated predictions for the future. A promising approach to this type of challenge is machine learning. In order to approach this machine learning problem methodically, the book \citetitle{geron_hands-machine_2019} by \citeauthor{geron_hands-machine_2019} is used as a guideline. The implemented methodology results in a \emph{Python Jupyter Notebook}, in which it is analysed whether and with which models this problem can be solved. In the first step, \citeauthor{geron_hands-machine_2019} advises to think fundamentally and to answer basic questions about the problem. This is done in the following section.

\subimport{section/}{1_business_context.tex}
\subimport{section/2_data/}{data.tex}
\subimport{section/}{3_models.tex}
\subimport{section/}{4_betting_odds_impact.tex}
\subimport{section/}{5_application.tex}



