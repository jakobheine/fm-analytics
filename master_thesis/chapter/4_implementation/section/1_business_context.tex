\section{Business Context}

Based on the rules explained in \autoref{chap:SPITCH}, two main findings emerge. The first point is the main objective of the game: assemble a team of 11 players that will score as many points as possible on the upcoming match day. The second point is that the constraint placed on the players, the budget, can be converted to a player's total score. Regarding the first point, it is necessary to distinguish at what point the main goal of the game is achieved. There are three different angles to approach this: if only the problem itself is considered, the goal would be to put together the team that scores highest. Secondly, from a game perspective, it would be enough to field the best team of all competitors, i.e., to place first in the final ranking. Lastly, a purely economic objective would be to put together a team that makes a profit by ending in the profit zone at the end of the game. Past rankings show that even the first place of a match day never achieves the full number of points. Instead, the top places only achieve 80 to 85\% of the maximum possible score. Furthermore, it is important to notice that there is no well-known manager among these top places who always achieve top positions. According to these observations, it looks pretty challenging to field the best possible team, let alone to achieve first place on a regular basis. A probable explanation for this is the strong influence of the luck factor due to things that are not predictable, such as injuries. Therefore, it seems much more realistic to pursue the latter, purely economic goal, to end in the winning zone regularly. The previous rankings reveal that the points needed to reach the winning zone in the staked fields are lower, between 60 and 65\% of the maximum possible score. For this reason, the aim of this work is to write a model that regularly achieves a score that corresponds to 65\% of the best possible score.


