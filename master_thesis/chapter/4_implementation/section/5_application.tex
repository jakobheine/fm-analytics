\clearpage \section{Application}

\label{chap:application}

This last section of the chapter aims to present the entire application that was developed in the course of this thesis. The source code for the application, as well as the Jupyter Notebook in which the data mining part was implemented, can be found as a \emph{Git} repository on \emph{GitHub} under the following link:

\begin{quote}
    \centering
    \underline{https://github.com/jakobheine/fm-analytics/tree/v.1.0.0}
\end{quote}

As already mentioned at the beginning of this chapter, the application is implemented in a microservice architecture. All services are created as \emph{Docker} containers, which makes them easy to deploy. The entire tool is deployed on a virtual private server (VPS) hosted on Hetzner. \parencite[][]{hetzner_about_2021} The following figure shows the architecture and how the microservices interact with each other.

\begin{figure}[H]
    \centering
    \fbox{\includegraphics[width=15cm]{chapter/4_implementation/figures/microservice-architecture.png}}
    \captionsetup{justification=centering}
    \caption{Microservice-Architecture of the Application}
    \label{fig:microservice-architecture}
\end{figure}

The green arrows indicate the primary process. In this process, the \emph{Crawler} collects the current data from \emph{SPITCH} and \emph{the odds-api} via the \emph{Proxy} and stores it in the \emph{Database}. This data is then queried by the \emph{Prediction} service, which predicts the individual scores of the players using the \emph{hyperparameter-tuned Random Forest Model} and forwards them to the \emph{Lineup} service. The \emph{Lineup} service creates the optimal line-up from the predicted scores and the transfer market values, which it then sends to the end-users via a \emph{Telegram Bot}. This process starts 15 minutes before the kick-off of a matchday so that the data is as up-to-date as possible, but the managers still have enough time to line up the team predicted by the tool. In addition, the manager can start the process himself at any time. Furthermore, the \emph{Backup} service creates a backup of the database every 24 hours. The following screenshot shows the messages the application is sending to its users.

\vskip 0.5cm

\begin{figure}[H]
    \centering
    \fbox{\includegraphics[width=12cm]{chapter/4_implementation/figures/telegram-message.png}}
    \captionsetup{justification=centering}
    \caption{Screenshot of the Applications Messages}
    \label{fig:screenshot-telegram}
\end{figure}

\clearpage To deploy the application, only a server with \emph{Docker}, \emph{Docker-Compose}, and \emph{Git} installed is needed. After the \emph{Github} repository has been cloned, the application can be started with the command:  \lstinline{docker-compose up}. \\
\indent With the help of \emph{Docker-Compose}, each service is started as a \emph{Docker} container. For this purpose, each \emph{Docker image} is initially built once. A \emph{Python Virtual Environment} is created in each service, which installs the previously defined dependencies. When the system is initially started, all services are created, and all data of the current season is stored in the \emph{Database}. If the system is shut down and restarted later, the old \emph{Database} is taken over, and the services only request the missing information.