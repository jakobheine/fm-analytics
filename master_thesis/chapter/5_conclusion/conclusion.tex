\chapter{Conclusion}

Fantasy Leagues flourish under the influence of the \emph{Big Data} age. The immense growth in data and statistics makes these games more complex than ever before, yet this growth offers a wide range of improvement potentials. In this thesis, it was investigated to what extent this data can be used in machine learning models which predict future player performances. Therefore, the fundamental research question asked was: \emph{'How accurately can individual soccer player performances be predicted using historical data?'}.

To answer this question, three steps had to be taken. The first step to answering this question was a literature review to overview possible solution approaches. This review revealed that no research has yet been done on precisely this kind of problem. Nevertheless, related research was found and applied, resulting in multiple feature and model proposals. One of these features was the betting odds, which raised the question of whether betting odds affect the predictions positively. Thus, the central research question could be updated into \emph{'How accurately can individual soccer player performances be predicted using historical data and betting odds?'}.

In the second step, the question at hand was translated into a machine learning problem. Through this translation, the goal could be set to write a model that regularly earns a prize in a Fantasy League competition by reaching at least a line-up score of 65\% of the best possible score. In order to write this model, a data basis had to be established first. An application was written to procure, process and, store the data.

\clearpage In the final step, using the data, several models, with and without betting odds, were compared with each other. From the comparison, two types of machine learning models emerged that performed best: \emph{Random Forest} and \emph{Multiple Linear Regression}. Hence, the thesis proves that these two types of models, which have been shown to solve similar problems in the past, also performed well on the issue at hand. On the downside, models, like \emph{Support Vector Machines} or \emph{Decision Trees}, that promised success as well did not offer accurate predictions. 

In the experiment on the test dataset, the Random Forest model won a prize twice in nine rounds and thus achieved the desired goal, winning a total prize of €33.86. On average, the model misjudged each player by about 100 points, which is quite a lot for an average score of 200 per player. Nevertheless, even this accuracy seems to be enough to win in the competition \emph{SPITCH} regularly. Another aspect this work proved in this way is that football is a challenging sport to predict, and therefore the entire competition offered by \emph{SPITCH} resembles many characteristics of gambling. In addition, the calculated \emph{Cohen's d} of 0.313 between the treatment and baseline models indicates that the betting odds indeed have a small but positive effect on the predictions, even though the calculated \emph{p-value} of 0.5212 implies an inconclusive sample. Finally, the central research question can be answered as follows: Individual player performances cannot be predicted to the exact point. There are far too many factors to consider, which are not or only very difficult to realise in a model. Nevertheless, machine learning models can help to assemble a line-up that is occasionally better than 75\% of the competitors.

In the future, to further improve the models, more features can be added. Examples would be the official transfer market value or characteristics about the playing style of different players or teams. In order to further validate the impact of betting odds and generally to increase the models' accuracies, more data can be collected and used. In addition, other forecast values can be examined, such as expected goals or spread bets.