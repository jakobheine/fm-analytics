\section{Fantasy Leagues}

Fantasy Leagues can look back on a history of over 60 years. Wilfred Winkenbach, a sports entrepreneur and enthusiast from the USA, designed a fantasy golf game in the 1950s. In this game, a team was made up of several golfers, and the team with the lowest swings in total won. Building on the success of this game, Winkenbach developed the first fantasy football league in 1962, which is similar to today's fantasy leagues. \parencite[cf.][]{green_wink_2014} This league consisted of 8 participants, friends or co-workers of Winkenbach, who met in a restaurant and wrote down their line-up for the coming season. The scoring system was kept very simple and was limited to the main events in a football game: touchdowns, field goals and interceptions. The simple reason was that each event had to be counted by hand by the game master. \parencite[cf.][]{fabiano_fantasy_2007} From this game, leagues quickly developed in other sports, such as baseball. One of the reasons this type of game first spread in the USA is the ease of assigning points to individual actions in the popular represented sports. For example, during an attack in American football, there are several plays, separated by pauses in which it can be assessed relatively clearly, for example, by the yards gained or lost, whether the play was successful. In soccer, on the other hand, there are fewer interruptions, plus unlike in baseball or American football, there are no intermediate milestones that can be reached between moves. These missing pauses lead to a more wild game, with difficult to evaluate actions. In addition, although there are roles within a soccer team, these roles, except the goalkeeper, are more strategic and do not restrict the players in their playing actions. A defender can score goals or intercept passes just as well as a striker. In contrast, in American football or baseball, each player often has one single task per turn that can either succeed or not. All these factors did sports like soccer challenging, if not impossible, to implement as fantasy leagues in the past.

However, with the advancement of modern image recognition technologies and player tracking devices such as two high-resolution cameras per playing side, these times are gone. \parencite[cf.][]{hoffmann_millionen_2014} Nowadays, every event on a soccer pitch is automatically trackable and therefore offers the possibility to evaluate the performance of different players much more accurately. These advances allow fantasy soccer leagues to exist, as they can build their game on this data basis.

Nevertheless, the main goal of fantasy leagues is always the same for all sports: assemble a team that performs best. However, there are differences between the fantasy leagues in how this performance is evaluated. One major factor for this difference originates in the differences in the sport disciplines themself. Despite that, even leagues in the same discipline can differ to create a unique selling point. Since this work is primarily focused on soccer, the following considerations are limited to fantasy soccer leagues. Furthermore, the decision was made to use the provider SPITCH, which only offered the first Bundesliga at the time of writing. That is why all comparisons are made concerning this game system. 

The list of differences between the individual fantasy soccer leagues is long. Each provider of a league wants to have its unique selling point and highlights different tactical elements. It is not the aim of this thesis to show all the differences. This section merely serves to give a rough overview of the world of Fantasy Soccer Leagues and, in addition, to show that each game must be approached strategically differently and, as a result, different questions must be asked. The research in this paper, therefore, applies primarily to the game SPITCH. Nevertheless, this thesis aims to shed light on problems as general as possible and be useful for research in similar areas. \\
A major difference in fantasy soccer is the national league in which the fantasy league is located. For example, there are providers for the English \emph{Barclays Premier League}, the Italian \emph{Serie A} and for the German \emph{Bundesliga}, the latter being observed in this work. However, it is not only the selected national league that differentiates the various providers. Furthermore, a differentiation can be made between the availability of players. For example, one popular game mode exists where 20 fantasy managers compete against each other, similar to the actual real-world competition. Each fantasy manager is assigned a team of random players. These own players can be traded with the other 19 teams for other players or money on a virtual transfer market. It is important to note that this game mode creates a segregated space, where the participants compete against each other continuously over an entire season. On top of that, each player exists \textbf{only once} and can only play for one fantasy manager's team at the time. In contrast, in SPITCH, any player can be bought and used by any fantasy manager. Furthermore, the participants do not play in a segregated space but with unlimited opponents. Further details and the general regulations are described in the following section.