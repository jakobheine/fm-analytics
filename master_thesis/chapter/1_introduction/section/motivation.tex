\section{Motivation}

\begin{quote}
    \begin{center}
        \emph{'Data is the new oil.'}   -- Clive Humby, 2006
    \end{center}
\end{quote}

What Clive Humby already recognised in 2006 is even more relevant today than ever before. Moreover, in the age of \emph{Big Data}, the utilisation of this data is probably even more essential. The ability to gain insights by analysing this data and thereby anticipate the future is an art in itself and is still being vigorously explored in science today. More and more, efforts are made to predict the future with the help of this data and computing techniques such as \emph{Artificial Intelligence} or \emph{Machine Learning}.

At the same time, the \emph{Big Data} age does not spare the world of sport. \parencite[cf.][]{rein_big_2016} In recent years, sports teams and associations have started to collect data during their games extensively. The teams analyse this data to keep improving, while the associations use the data to provide a more comprehensive gaming experience for their viewers. In almost every sport, this movement also led to a new sub-discipline of these sports: the Fantasy Leagues. Already started in 1962 for American football, this segment experienced a boom due to \emph{Big Data}, as every sport could now be played as Fantasy League with an increasing complexity thanks to more accurate tracking possibilities.

\clearpage Thereby, the technique of successfully analysing the data and predicting the future with machine learning models is enjoyable for all participants of the sport: the teams know where they can improve, the coach understands which tactics or players work best, and the supporters are even more involved in the sport. Above all, the Fantasy League organisers nowadays use these analyses to know how they have to design their game to remain as exciting as possible. Mainly because these models always try to look into the future, the idea of incorporating a metric that is supposed to represent the future is close at hand. An example of these metrics is a prediction market, which dares to look into the future with the help of swarm intelligence. There are different types of prediction markets in sports, ranging from spread betting to betting odds. This industry is even older than Fantasy Leagues, and not at all surprisingly, it is booming with the growth of data and analysis in sports. In fact, Fantasy Leagues are nothing more than more complex sports bets. 

Although all mentioned areas are thriving, there still exists almost no scientific literature on their unification. This disregard leads to wasted potentials, as the stated benefits are numerous. Furthermore, the data required is already available and reliable. For these reasons, this thesis intends to examine one particular area of this unification to see how far machine learning can be used in this context to predict the future successfully. 