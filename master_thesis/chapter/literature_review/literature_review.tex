\chapter{Literature Review}
% what is going on in this chapter (0,25p)
This chapter aims to present the current state of research in two different domains. The first is about predicting sporting events using machine learning. The latter examines sports betting with a particular focus on betting odds and how these can help to predict events in the future. \\
The established guidelines of \cite{vom_brocke_standing_2015} and \cite{webster_guest_2002}, were used to determine the research status and respectively document the literature search process. As stated by \citeauthor{webster_guest_2002}, two types of literature reviews exist. This literature review belongs to the second type, which is, according to \citeauthor{webster_guest_2002}, in general, shorter and where \emph{'authors [...] tackle an emerging issue that would benefit from exposure to potential theoretical foundations'} \parencite[, p. 14]{webster_guest_2002}. First, as recommended by \citeauthor{vom_brocke_standing_2015}, the literature search process is documented as accurately as possible to facilitate future research on this topic. Then, the literature found is summarised in a concept matrix according to \citeauthor{webster_guest_2002} and examined according to specially selected criteria. On this basis, research gaps get identified, and finally, the research question for this thesis gets formulated.

% describing the search process
According to \citeauthor{vom_brocke_standing_2015}, in order to find relevant literature on the research areas dealt with in the thesis, the topic must first be divided into individual concepts. These concepts help to find literature in scholarly databases using keyword search. The keywords searched for in this thesis were \emph{'fantasy football'}, \emph{'machine learning'}, \emph{'prediction'} and \emph{'betting odds'}. The keywords were entered in every existing constellation to find articles that do not correspond to all keywords. Based on the research of \cite{gusenbauer_google_2019}, \emph{Google Scholar} and \emph{Microsoft Academic}, the most extensive academic search engines were used for the literature search. When selecting the results from this search, attention was paid to the currently awarded VHB journal rankings \parencite[see][]{vhb_e_v_vhb-jourqual3_2015} for the sub-field of business informatics to ensure that the literature researched is of high quality. One journal that would be less considered following this approach, but seems extremely relevant to the research in this thesis, is the \emph{Journal of Quantitative Analysis in Sports} (JQAS). This journal gets published by the American Statistical Association (ASA), which according to themselves, \emph{'is the world's largest community of statisticians'} \parencite[see][]{noauthor_about_nodate}. Using the papers from the JQAS and journals highly ranked by the VHB, the remaining literature got found using backward search and forward search suggested by \citeauthor{webster_guest_2002}.



% explaining the concept matrix and the criteria (1p)

% conclude the concepts from the papers
% (3p)

% give research gap, even for future stuff WW

% explain chosen gap / research question (1p)

% overall ~7 pages