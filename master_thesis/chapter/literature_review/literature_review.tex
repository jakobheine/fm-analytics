\chapter{Literature Review}
% what is going on in this chapter (0,25p)
This chapter aims to present the current state of research in two different domains. The first is about predicting sporting events using machine learning. The latter examines the field of sports betting with a particular focus on betting odds and how they can be used to predict events in the future. \\
The established guidelines of \cite{vom_brocke_standing_2015} and \cite{webster_guest_2002}, were used to determine the research status and respectively document the literature search process. As stated by \citeauthor{webster_guest_2002}, two types of literature reviews exist. This literature review belongs to the second type, which is, according to \citeauthor{webster_guest_2002}, in general, shorter and where \emph{'authors [...] tackle an emerging issue that would benefit from exposure to potential theoretical foundations'} \parencite[, p. 14]{webster_guest_2002}. First, as recommended by \citeauthor{vom_brocke_standing_2015}, the literature search process is documented as accurately as possible to facilitate future research on this topic. Then, the literature found is summarised in a concept matrix according to \citeauthor{webster_guest_2002} and examined according to specially selected criteria. On this basis, research gaps get identified, and finally, the research question for this thesis gets formulated.

% describing the search process, start with why it is important BROCKE (0,5p)
% which journals -> ranking, how close they are to the topic
% which database -> source for most relevant
% which articles → matrix next point
% backward, forward
%     (1p)

% explaining the concept matrix and the criteria, NAME HOW MANY RESOURCES YOU WATCHED
% (1p)

% conclude the concepts from the papers
% (3p)

% give research gap, even for future stuff WW

% explain chosen gap / research question (1p)

% overall ~7 pages