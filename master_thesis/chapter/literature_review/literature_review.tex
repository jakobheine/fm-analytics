\chapter{Literature Review}
% what is going on in this chapter (0,25p)
This chapter presents the current state of research in two different domains. The first is about predicting sporting events using machine learning. The latter examines sports betting with a particular focus on betting odds and how these can help to predict events in the future. \\
\indent The established guidelines of \cite{vom_brocke_standing_2015} and \cite{webster_guest_2002}, are used to determine the current state of research and respectively document the literature search process. As stated by \citeauthor{webster_guest_2002}, two types of literature reviews exist. This literature review belongs to the second type, which is, according to \citeauthor{webster_guest_2002}, in general, shorter and where \emph{'authors [...] tackle an emerging issue that would benefit from exposure to potential theoretical foundations'} \parencite[, p. 14]{webster_guest_2002}. First, as recommended by \cite{vom_brocke_standing_2015}, the literature search process is documented as accurately as possible to facilitate future research on this topic. Then, the literature found is summarised in a concept matrix according to \cite{webster_guest_2002} and examined according to specially selected criteria. Based on this examination, research gaps are identified, and finally, the research question for this thesis is formulated.

% describing the search process
According to \cite{vom_brocke_standing_2015}, in order to find relevant literature on the research areas dealt with the thesis, the topic is divided into separate concepts. These concepts help to find literature in scholarly databases using keyword search. The keywords searched for in this thesis were \emph{'fantasy football'}, \emph{'machine learning'}, \emph{'prediction'} and \emph{'betting odds'}. The keywords were entered in every existing combination to find articles that do not correspond to all keywords. Based on the research of \cite{gusenbauer_google_2019}, \emph{Google Scholar} and \emph{Microsoft Academic}, the most extensive academic search engines were used for the literature search. When selecting the results from this search, attention is paid to the currently awarded VHB journal rankings \parencite[see][]{vhb_e_v_vhb-jourqual3_2015} for the sub-field of business informatics to ensure that the literature researched is of high quality. This ranking is chosen because it is well-known and accepted in the German research area. One journal that would be less considered following this ranking, but seems extremely relevant to the research in this thesis, is the \emph{Journal of Quantitative Analysis in Sports} (JQAS). This journal gets published by the American Statistical Association (ASA), which according to themselves, \emph{'is the world's largest community of statisticians'} \parencite[see][]{noauthor_about_nodate}. Using the papers from the JQAS and journals highly ranked by the VHB, the remaining literature got found using backward search and forward search suggested by \citeauthor{webster_guest_2002}.

% explaining the concept matrix and the criteria (1p)
In the process mentioned above, 22 papers were examined and compared in a concept matrix as required by \citeauthor{webster_guest_2002}. Due to its size and for the sake of readability, this matrix is in the appendix. Nevertheless, the concepts used to examine the papers will be briefly discussed from left to right in this paragraph. \\
\indent The year of publication, the VHB ranking and the distinction in which form the paper was published serve to evaluate the quality of the literature. That is to ensure that primarily the most recent papers in renowned peer-reviewed journals were analysed. The sport discipline helps to notice similar approaches in different sports. While sports differ, some are more related than others. The main idea behind this is that there may be viable approaches from a similar sport that would have been unconsidered otherwise.  \\
\indent During the research, to the best of my knowledge, no publication was found which deals precisely with the problem at hand. For this reason, the research had to focus on similar approaches, objectives or tasks. The solving approaches vary from more straightforward approaches such as mixed integer programming to more complex multi-hierarchical Bayesian models. Some publications used a combination of several methodologies, which are strongly dependent on the task to be solved. A distinction was therefore made between optimisation and prediction tasks. Although almost all papers unanimously had the goal of setting up a team that would score as many points as possible, they came at the solution differently. The matrix distinguishes between publications that optimised only the team performance as a whole and those that predicted the performance for each individual player and then combined the best players into a team. At the same time, it investigated which papers relied on betting odds or another form of prediction markets. Lastly, the data used in each publication was analysed. Due to the always different data, a generalised view was applied, which examines whether time-series data is used, whether the home advantage was taken into account and whether betting odds were used.

% conclude the concepts from the papers (3p)
\textbf{Placeholder: conclude the concepts from the papers}

% explain chosen gap / research question (1p)
\textbf{Placeholder: explain chosen gap / research question }


% CONCEPT-MATRIX-STUFF
% \settowidth{\rotheadsize}{Publication}

% \begin{longtable}{lcccccccc}\toprule
%              &                  & \multicolumn{3}{c}{Published In} & \\ \cmidrule{3-5}
% Paper        & \rothead[c]{Publication\\Year} & \rothead[c]{Journal} & \rothead[c]{Conference\\Proceeding} & \rothead[c]{Other} & \rothead[c]{VBA-\\Ranking} & \rothead[c]{Discipline} \\ \midrule
% Tool Alpha   & NO               & OK      & a\\
% Tool Delta   & NO               & NO \\
% Tool Gamma   & NO               & OK \\
% Tool Theta   & NO               & NO \\
% Tool Upsilon & OK               & NO \\
% \textbf{Tool X (our proposal)} & OK &   OK \\ \bottomrule
% \end{longtable}